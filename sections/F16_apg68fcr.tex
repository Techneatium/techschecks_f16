\chapter{APG-68 FCR}
\thumbtab{APG-68 FCR}{2}
\localtableofcontents
\cleardoublepage

\section{OVERVIEW}
% \begin{table}[h]
%     \centering
%     \caption{\textbf{Overview of APG-68 A-A Radar Modes}}
%     \label{tab:apg68overviewaa}
%     \begin{tabular}{c c | c | c | c | c | c}
%         \toprule
%         \multicolumn{3}{c |}{\blue{CRM}} & \multicolumn{4}{c}{\blue{ACM}} \\
%         \multicolumn{3}{c |}{\Cref{subsec:crm}} & \multicolumn{4}{c}{\Cref{subsec:acm}} \\
%         \midrule
%         \multicolumn{2}{c |}{\textbf{RWS}} & \textbf{TWS} & \textbf{BORE} & \textbf{Vertical} & \textbf{HUD} & \textbf{Slewable} \\
%         SAM & DTT & & & & & \\
%         \midrule
%         \multicolumn{7}{c}{\blue{STT}} \\
%         \multicolumn{7}{c}{\Cref{subsec:stt}} \\
%         \bottomrule
%     \end{tabular}
% \end{table}

\begin{tcoloritemize}
    \blueitem{A-A Modes}{
    \begin{subitemize}
        \item \textbf{CRM} --- \textbf{C}ombined \textbf{R}adar \textbf{M}ode, \break see \Cref{subsec:crm}
        \begin{itemize}
            \item \textbf{RWS} --- \textbf{R}ange \textbf{W}hile \textbf{S}earch, \break see \Cref{subsec:rws}
            \item \textbf{TWS} --- \textbf{T}rack \textbf{W}hile \textbf{S}can, \break see \Cref{subsec:tws}
        \end{itemize}
        \item \textbf{ACM} --- \textbf{A}ir \textbf{C}ombat \textbf{M}ode, see \Cref{subsec:acm}
        \item \textbf{STT} --- \textbf{S}ingle \textbf{T}arget \textbf{T}rack, see \Cref{subsec:stt}
    \end{subitemize}}
    \blueitem{A-G Modes}{\textbf{Work In Progress}
    \begin{subitemize}
        \item \textbf{GM} --- \textbf{G}round \textbf{M}ap
        \item \textbf{GMT} --- \textbf{G}round \textbf{M}oving \textbf{T}arget
    \end{subitemize}}
\end{tcoloritemize}

\begin{figure}[htbp]
    \centering
    \begin{tikzpicture}[auto, node distance=10mm,x=1mm, y=1mm, very thick,
        >={Latex[round]}
        ]
        
        % \node[<options>](<coordinate name>)at(<coordinate>){<text>};
        \node[
            hyperref node=subsec:stt,
            rectangle, 
            rounded corners,
            minimum width=90mm,
            minimum height=7.5mm,
            draw,
        ](stt)at(0,0){\blue{STT}--- \Cref{subsec:stt}};
        \node[
            hyperref node=subsec:crm,
            rectangle,
            rounded corners,
            minimum width=25mm,
            minimum height=7.5mm,
            draw,
        ](crm)at(0,30){\begin{tabular}{c}\blue{CRM}\\ \Cref{subsec:crm}\end{tabular}};
        \node[
            hyperref node=subsec:rws,
            rectangle,
            rounded corners,
            minimum width=25mm,
            minimum height=7.5mm,
            draw,
        ](rws)at(0,15){\begin{tabular}{c}\blue{RWS}\\ \Cref{subsec:rws}\end{tabular}};
        \node[
            hyperref node=subsec:tws,
            rectangle,
            rounded corners,
            minimum width=25mm,
            minimum height=7.5mm,
            draw,
        ](tws)at(32.5,15){\begin{tabular}{c}\blue{TWS}\\ \Cref{subsec:tws}\end{tabular}};
        \node[
            hyperref node=subsec:acm,
            rectangle,
            rounded corners,
            minimum width=25mm,
            minimum height=7.5mm,
            draw,
        ](acm)at(0,-30){\begin{tabular}{c}\blue{ACM}\\ \Cref{subsec:acm}\end{tabular}};
        \node[
            rectangle,
            rounded corners,
            minimum width=25mm,
            minimum height=7.5mm,
            draw,
        ](bore)at(0,-15){\textbf{BORE}};
        \node[
            rectangle,
            rounded corners,
            minimum width=25mm,
            minimum height=7.5mm,
            draw,
        ](hud)at(32.5,-30){\textbf{HUD}};
        \node[
            rectangle,
            rounded corners,
            minimum width=25mm,
            minimum height=7.5mm,
            draw,
        ](vert)at(0,-45){\textbf{Vertical}};

        % Lines
        \draw [->]
            (crm) -- (rws);
        \draw [->, rounded corners]
            (crm) -| (tws);
        \draw [->]
            let
                \p1=(rws.south),
                \p2=(stt.north),
            in
                (\p1) -- (\x1,\y2);
        \draw [->]
            let
                \p1=(tws.south),
                \p2=(stt.north),
            in
                (\p1) -- (\x1,\y2);
        \draw [->]
            (acm) -- (bore);
        \draw [->]
            (acm) -- (hud);
        \draw [->]
            (acm) -- (vert);
        \draw [->]
            let
                \p1=(bore.north),
                \p2=(stt.south),
            in
                (\p1) -- (\x1,\y2);
        \draw [->]
            let
                \p1=(hud.north),
                \p2=(stt.south),
            in
                (\p1) -- (\x1,\y2);
        \draw [->, rounded corners]
            let
                \p1=(vert.west),
                \p2=(stt.south),
            in
                (\p1) -- (\x1-22.5mm,\y1) -- (\x1-22.5mm,\y2);
                
    \end{tikzpicture}
    \caption{\textbf{A-A Radar Modes Overview}}
\end{figure}

\clearpage

\section{AIR-TO-AIR MODES}

\subsection{CRM}
\label{subsec:crm}

\begin{tcoloritemize}
    \blueitem{CRM}{
        \textbf{C}ombined \textbf{R}adar \textbf{M}ode --- Combines A-A search submodes:

        \begin{subitemize}
            \item \textbf{RWS} --- \textbf{R}ange \textbf{W}hile \textbf{S}earch,
            see \Cref{subsec:rws}
            \item \textbf{TWS} --- \textbf{T}rack \textbf{W}hile \textbf{S}can,
            see \Cref{subsec:tws}
        \end{subitemize}

        Selected by default on FCR start-up
    }
    \blueitem{Change CRM \break Submode}{
    \begin{subitemize}
        \item \textbf{OSB 2} \dotfill \textbf{Press}
        \item Or \textbf{TMS} \dotfill \textbf{Right Long}
    \end{subitemize}}
\end{tcoloritemize}

\begin{figure}[htbp]
    \centering
    \begin{tikzpicture}[auto, node distance=10mm,x=1mm, y=1mm, ultra thick,
        >={Latex[round]}
        ]
        
        % \node[<options>](<coordinate name>)at(<coordinate>){<text>};
        \node[
            hyperref node=subsec:rws,
            diamond,
            draw,
        ](rws)at(0,0){\textbf{RWS}};
        \node[
            rectangle,
            rounded corners,
            minimum width=20mm,
            minimum height=7.5mm,
            draw,
        ](sam)at(0, 50){\textbf{SAM}};
        \node[
            rectangle,
            rounded corners,
            minimum width=20mm,
            minimum height=7.5mm,
            draw,
        ](dtt)at(-30,65){\textbf{DTT}};
        \node[
            diamond,
            minimum width=15mm,
            minimum height=15mm,
            draw,
        ](tws)at(30,0){\textbf{TWS}};
        \node[
            rectangle,
            rounded corners,
            minimum width=20mm,
            minimum height=7.5mm,
            draw,
            dashed,
        ](systgt)at(30,20){\textbf{System Tgt}};
        \node[
            rectangle,
            rounded corners,
            minimum width=20mm,
            minimum height=7.5mm,
            draw,
            dashed,
        ](cursortgt)at(30,35){\textbf{Cursor Tgt}};
        \node[
            rectangle,
            rounded corners,
            minimum width=20mm,
            minimum height=7.5mm,
            draw,
            dashed,
        ](bug)at(30,50){\textbf{Bugged}};
        \node[
            rectangle, 
            rounded corners,
            minimum width=80mm,
            minimum height=7.5mm,
            draw, 
        ](stt)at(0,80){\textbf{STT}};
        
        % Lines
        \draw [->, thick]
            (rws) -- node[pos=0.5, left]{\scriptsize\textbf{TMS FWD}} (sam);
        \draw [->, rounded corners, thick]
            (sam) -| node[pos=0.25, above]{\scriptsize\textbf{TMS FWD}}node[pos=0.25, below]{\scriptsize\textbf{(2nd Tgt)}} (dtt);
        \draw [->, thick]
            let
                \p1=(dtt.north),
                \p2=(stt.south)
            in
                (\p1) -- node[pos=0.5, left]{\scriptsize\textbf{TMS FWD}} (\x1,\y2);
        \draw [->, thick]
            let
                \p1=(sam.north),
                \p2=(stt.south)
            in
                (\p1) -- node[pos=0.5, left]{\scriptsize\textbf{TMS FWD}}node[pos=0.5, right]{\scriptsize\textbf{(1st Tgt)}} (\x1,\y2);
        \draw [<->, thick]
            (rws) -- node[pos=0.5, above]{\scriptsize\textbf{TMS RIGHT}}node[pos=0.5, below]{\scriptsize\textbf{(long)}} (tws);
        \draw[->, thick]
            (tws) -- node[pos=0.5, left]{\scriptsize\textbf{TMS FWD}} (systgt);
        \draw[->, thick]
            (systgt) -- node[pos=0.5, left]{\scriptsize\textbf{Cursor Over}} (cursortgt);
        \draw[->, thick]
            (cursortgt) -- node[pos=0.5, left]{\scriptsize\textbf{TMS FWD}} (bug);
        \draw [->, thick]
            let
                \p1=(bug.north),
                \p2=(stt.south)
            in
                (\p1) -- node[pos=0.5, left]{\scriptsize\textbf{TMS FWD}} (\x1,\y2);

        % Brackets
        \draw [decorate, decoration={brace}, ultra thick] 
            let
                \p1 = (stt.north),
                \p2 = (stt.east),
                \p3 = (sam.south),
            in
                (\x2+12mm,\y1) -- (\x2+12mm,\y3);
        \path
            let
                \p1 = (stt.north),
                \p2 = (stt.east),
                \p3 = (sam.south),
            in
                (\x2+14mm,\y1) -- node[pos=0.5, right, rotate=90, anchor=north]{\small\textbf{Supports AIM-120C Launch}} (\x2+14mm,\y3);
        
        \draw [decorate, decoration={brace}, ultra thick] 
            let
                \p1 = (dtt.north),
                \p2 = (stt.east),
                \p3 = (sam.south),
            in
                (\x2+3mm,\y1) -- (\x2+3mm,\y3);
        \path
            let
                \p1 = (dtt.north),
                \p2 = (stt.east),
                \p3 = (sam.south),
            in
                (\x2+5mm,\y1) -- node[pos=0.5, right, rotate=90, anchor=north]{\small\textbf{No Launch Warning}} (\x2+5mm,\y3);
    \end{tikzpicture}
    \caption{\textbf{CRM Radar Modes Overview}}
    \label{fig:crmoverview}
\end{figure}

\clearpage

\subsubsection{A-A SEARCH SCAN PATTERN}

\begin{tcoloritemize}
    \blueitem{Radar Gimbal Limits}{
    APG-68 is mechanically scanned on 2 axis gimbal

    \begin{subitemize}
        \item \textbf{Azimuth} --- \pm60 deg (120 deg coverage)
        \item \textbf{Vertical} --- \pm60 deg (120 deg coverage)
    \end{subitemize}}
    \blueitem{Azimuth Scan Pattern}{
    \begin{subitemize}
        \item \textbf{Azimuth search patterns}
        \begin{itemize}
            \item \textbf{A6} --- \pm60 deg
            \item \textbf{A3} --- \pm30 deg (centered on cursor)
            \item \textbf{A1} --- \pm10 deg (centered on cursor)
        \end{itemize}
        \item \textbf{Cycled via}
        \begin{itemize}
            \item \textbf{OSB 18} cycles available azimuth patterns
            \item Walking cursor off side of FCR display in RWS cycles A6/A3
        \end{itemize}
        \item \textbf{TWS \& DTT modes use a special \pm25 deg, 3 bar scan pattern}
        \begin{itemize}
            \item \textbf{A2} --- \pm25 deg (centered on cursor / target)
        \end{itemize}
    \end{subitemize}}
    \blueitem{Bar Scan Pattern}{
    Elevation scan volume controlled by selecting number of ``bars'' (horizontal sweeps)
    \begin{subitemize}
        \item \textbf{B4} / \textbf{B2} / \textbf{B1} --- cycled via \textbf{OSB 17}
        \item \textbf{B3} --- selected by TWS \& DTT modes
    \end{subitemize}}
\end{tcoloritemize}

\begin{figure}[htbp]
    \centering
    \fbox{
    \begin{minipage}[t][75mm][t]{100mm}
        \center{\large\textbf{Radar Azimuth Scan}}
        \begin{itemize}
            \item Top-down view of ``pie'' diagram
            \item Show various scan patterns as smaller slices of ``pie''
            \begin{itemize}
                \item maybe all centered?
                \item maybe next to each other?
            \end{itemize}
        \end{itemize}
    \end{minipage}
    }
    \caption{FCR azimuth scan \& limits}
\end{figure}

\begin{figure}[htbp]
    \centering
    \fbox{
    \begin{minipage}[t][75mm][t]{100mm}
        \center{\large\textbf{Radar Elevation Scan}}
        \begin{itemize}
            \item Side-on view of ``pie'' diagram
            \item Show various scan patterns as smaller slices of ``pie''
            \begin{itemize}
                \item maybe on top of each other?
            \end{itemize}
            \item Show bar scan patterns as classic ``snake'' diagrams 
            \begin{itemize}
                \item maybe place immediately next to matching ``pie'' slice?
            \end{itemize}
        \end{itemize}
    \end{minipage}
    }
    \caption{FCR elevation scan \& limits}
\end{figure}

\subsubsection{FCR MFD CONTROLS}

\begin{tcoloritemize}
    \blueitem{Radar Mode}{
    \textbf{OSB 1} opens page allowing selection of radar mode

    \begin{subitemize}
        \item \textbf{Left Side (OSB 19-20)} --- A-A Modes
        \begin{itemize}
            \item CRM --- see \cref{subsec:crm}
            \item ACM --- see \cref{subsec:acm}
        \end{itemize}
        \item \textbf{Right Side (OSB 6-9)} --- A-G Modes
        \begin{itemize}
            \item See \cref{sec:fcr-ag}
        \end{itemize}
        \item \textbf{STBY (OSB 10)} --- places FCR in standby
    \end{subitemize}}
    \blueitem{Field of View}{
    \textbf{OSB 3} cycles field of view

    \begin{subitemize}
        \item \textbf{EXP} --- expanded field of view
        \item \textbf{NORM} --- normal field of view
    \end{subitemize}

    HOTAS \textbf{EXPAND/FOV} switch also cycles FOV}
    \blueitem{Override}{Pressing \textbf{OVRD (OSB 4)}  places FCR in standby}
    \blueitem{FCR Control}{\textbf{CNTL (OSB 5)} opens FCR control page}
    \blueitem{Datalink Mode}{\textbf{OSB 6} cycles datalink operating mode (WIP)}
    \blueitem{Declutter}{\textbf{DCLT (OSB 11)} opens FCR declutter page, allowing pilot to deselect symbology elements}
    \blueitem{Elevation Bar \break Select}{\textbf{OSB 17} cycles elevation bar search pattern}
    \blueitem{Azimuth Select}{
    \textbf{OSB 18} cycles azimuth search pattern}
    \blueitem{Range Select}{
    \textbf{OSB 19-20} adjusts FCR display range scale
    
    \begin{subitemize}
        \item \textbf{Can be cycled by ``walking'' cursor off top/bottom of display}
    \end{subitemize}
    }
\end{tcoloritemize}

\notebox{
    \textbf{Azimuth and range can also be cycled by ``walking'' acquisition cursor of side of FCR display}
}

\begin{figure}[htbp]
    \centering
    \fbox{
    \begin{minipage}[t][75mm][t]{100mm}
        \center{\large\textbf{MFD --- FCR --- RWS (Default CRM Page)}}
        \begin{itemize}
            \item Basic RWS search page with returns
            \item Several RWS radar returns
            \item Maybe steerpoint (wedding cake)
        \end{itemize}
    \end{minipage}
    }
    \caption{RWS MFD Symbology}
\end{figure}

\clearpage

\subsection{RWS}
\label{subsec:rws}
\begin{tcoloritemize}
    \blueitem{RWS}{
    \textbf{R}ange \textbf{W}hile \textbf{S}earch

    \begin{subitemize}
        \item \textbf{Fast, Long(er)-Range Search}
        \begin{itemize}
            \item Radar scans selectable search volume
            \item Displays raw target position returns
        \end{itemize}
        \item \textbf{\underline{No} track data} 
        \begin{itemize}
            \item exact target range, velocity, angle, etc.
            \item but can transition to advanced modes
        \end{itemize}
        % (exact target range, velocity, angle, etc.)
        % --- but can transition to advanced modes
    \end{subitemize}}
\end{tcoloritemize}

\begin{figure}[htbp]
    \centering
    \fbox{
    \begin{minipage}[t][75mm][t]{100mm}
        \center{\large\textbf{MFD --- FCR --- RWS Search}}
        \begin{itemize}
            \item Basic RWS search page with returns
            \item Several RWS radar returns
            \item Maybe steerpoint (wedding cake)
        \end{itemize}
    \end{minipage}
    }
    \caption{RWS MFD Symbology}
\end{figure}

\begin{tcoloritemize}
    \blueitem{SAM \break Submode}{
    \textbf{S}ituational \textbf{A}warness \textbf{M}ode

    \begin{subitemize}
        \item \textbf{Target is ``Bugged''} (Pseudo-Track)
        \begin{itemize}
            \item Can guide AIM-120C (w/o STT Lock)
            \item DLZ displayed if missile selected
        \end{itemize}
        \item \textbf{RWS search  continues}
        \begin{itemize}
            \item Scan pauses on SAM target
            \item FCR manages scan volume
        \end{itemize}
    \end{subitemize}}
    \blueitem{DTT \break Submode}{
    \textbf{D}ual \textbf{T}arget \textbf{T}rack

    \begin{subitemize}
        \item \textbf{2 Targets ``Bugged''} --- primary / secondary
        \begin{itemize}
            \item Can guide AIM-120C on \underline{primary target}
            \item DLZ displayed if missile selected
        \end{itemize}
        \item \textbf{TMS Left --- swaps primary / secondary}
        \item \textbf{RWS search  continues}
        \begin{itemize}
            \item Scan pauses on both primary / secondary
            \item FCR manages scan volume
        \end{itemize}
        \item \textbf{Within 10nm search pattern inhibited} --- radar only scans primary/secondary targets
        \item \textbf{Once \underline{either} target within 3nm automatically transitions to STT}
    \end{subitemize}}
    \blueitem{Spotlight}{
    \begin{subitemize}
        \item \textbf{Narrow scan centered on acquisition cursor}
        \begin{itemize}
            \item 4 bar elevation 
            \item \pm10 deg azimuth
        \end{itemize}
        \item \textbf{Useful to rapidly acquire radar returns from target at known position}
        \item \textbf{Activated by holding TMS Forward >1 sec}
        \begin{itemize}
            \item Enters SAM submode if target detected under Acquisition Cursor
            \item Returns to previous search pattern if no target under Acquisition Cursor when released
        \end{itemize}
    \end{subitemize}}
\end{tcoloritemize}

\marginfigeometry

\subsubsection{SELECT RWS MODE}
\begin{checklistenumerate}
    \blueitem{FCR Switch}{\textbf{FCR}}
    \blueitem{Desired MFD}{\textbf{FCR Page}, verify \textbf{SOI}}
    \blueitem{Radar Mode}{
        \textbf{CRM} (default), verify

        \begin{subitemize}
            \item \textbf{Dogfight/Missile Override} --- \textbf{NORM}
            \item \textbf{Radar Mode (OSB 1)} --- shows \textbf{CRM}
        \end{subitemize}
    }
    \blueitem{CRM Submode}{
        \textbf{RWS} (default), cycle via 

        \begin{subitemize}
            \item \textbf{TMS Right (long)} / \textbf{OSB 2} 
        \end{subitemize}
    }
\end{checklistenumerate}

\subsubsection{SAM / DTT ACQUISITION}
\begin{checklistenumerate}
    \blueitem{Locate Targets}{
    \begin{subenumerate}
        \item Correlate onboard/offboard sensors
        \begin{itemize}
            \item raw radar returns, RWR pings
            \item AWACS calls, datalink targets
        \end{itemize}
        \item Place targets within radar scan volume
    \end{subenumerate}}
    \blueitem{SAM Acquisition}{
    \marginpar{
        \captionsetup{type=figure}
        \fbox{
            \begin{minipage}[t][30mm][t]{\marginparwidth}
                \center{\textbf{Bugged Target Symbology}}
                \begin{itemize}[leftmargin=1em]
                    \item show bugged tgt 
                    \item maybe also non-bugged target for reference?
                \end{itemize}
            \end{minipage}
        }
        \caption{Bugged Target}
    }
    \begin{subenumerate}
        \item \textbf{Target} \dotfill under Acquisition Cursor
        \item \textbf{TMS} \dotfill \textbf{Forward (hold)}
        \item \textbf{Target} \dotfill Verify \textbf{Bugged}
    \end{subenumerate}
    \begin{subitemize}
        \item Can guide AIM-120C on Bugged target
        \item DLZ displayed if missile selected
        \item RWS search continues
    \end{subitemize}}
    \blueitem{DTT Acquisition}{(if desired)
    \marginpar{
        \captionsetup{type=figure}
        \fbox{
            \begin{minipage}[t][30mm][t]{\marginparwidth}
                \center{\textbf{DTT Target Symbology}}
                \begin{itemize}[leftmargin=1em]
                    \item show both DTT target symbols
                    \item maybe with cursor on one?
                \end{itemize}
            \end{minipage}
        }
        \caption{DTT Target}
    }
    \begin{subenumerate}
        \item \textbf{Target 2} \dotfill under Acquisition Cursor
        \item \textbf{TMS} \dotfill \textbf{Forward (hold)}
    \end{subenumerate}

    To swap primary / secondary target

    \begin{subenumerate}[start=3]
        \item \textbf{TMS} \dotfill \textbf{Left}
    \end{subenumerate}
    }
    \blueitem{STT Lock}{(if desired)
    \begin{subenumerate}
        \item \textbf{Target} \dotfill under Acquisition Cursor
        \item \textbf{TMS} \dotfill \textbf{Forward}
    \end{subenumerate}}
\end{checklistenumerate}

\subsubsection{SPOTLIGHT ACQUISITION}
\begin{checklistenumerate}
    \blueitem{Locate Targets}{
    \begin{subenumerate}
        \item Correlate onboard/offboard sensors
        \begin{itemize}
            \item raw radar returns, RWR pings
            \item AWACS calls, datalink targets
        \end{itemize}
        \item Place target within radar scan limits
    \end{subenumerate}}
    \blueitem{Spotlight Search}{
    \marginpar{
        \captionsetup{type=figure}
        \fbox{
            \begin{minipage}[t][60mm][t]{\marginparwidth}
                \center{\textbf{Spotlight Scan Symbology}}
                \begin{itemize}[leftmargin=1em]
                    \item show spotlight azimuth bars
                    \item maybe show both with raw rws return and once acquired?
                \end{itemize}
            \end{minipage}
        }
        \caption{Spotlight Search}
    }
    \begin{subenumerate}
        \item \textbf{Target} \dotfill near Acquisition Cursor
        \item \textbf{TMS} \dotfill \textbf{Forward (hold)}
        \begin{itemize}
            \item FCR enters \pm10deg scan around acquisition cursor
        \end{itemize}
    \end{subenumerate}}
    \blueitem{SAM Acquisition}{
    \begin{subenumerate}
        \item \textbf{Target} \dotfill under Acquisition Cursor
        \item \textbf{TMS} \dotfill \textbf{Forward (release)}
        \item \textbf{Target} \dotfill Verify \textbf{Bugged}
    \end{subenumerate}
    \begin{subitemize}
        \item Can guide AIM-120C on Bugged target
        \item DLZ displayed if missile selected
        \item RWS search continues
    \end{subitemize}}
    \blueitem{STT Lock}{(if desired)
    \begin{subenumerate}
        \item \textbf{Target} \dotfill under Acquisition Cursor
        \item \textbf{TMS} \dotfill \textbf{Forward}
    \end{subenumerate}}
\end{checklistenumerate}

\marginfigrestore

\clearpage

\subsection{TWS}
\label{subsec:tws}
\begin{tcoloritemize}
    \blueitem{TWS}{
    \textbf{T}rack \textbf{W}hile \textbf{S}can

    \begin{subitemize}
        \item \textbf{Multi-target tracking mode}
        \begin{itemize}
            \item Allows multi-target AIM-120 engagement
            \item Targets not locked --- \underline{no RWR warning}
        \end{itemize}
        \item \textbf{FCR builds trackfiles for each target}
        \begin{itemize}
            \item Predicts target movement between scans
            \item Slow update rate --- target maneuvers can cause radar to lose track
        \end{itemize}
        \item \textbf{Trackfiles can be in several states}
        \begin{itemize}
            \item Search / Track / System / Cursor / Bugged
        \end{itemize}
    \end{subitemize}}
\end{tcoloritemize}

\begin{figure}[htbp]
    \centering
    \fbox{
    \begin{minipage}[t][75mm][t]{100mm}
        \center{\large\textbf{MFD --- FCR --- TWS Search --- Cursor Target}}
        \begin{itemize}
            \item Basic TWS search page with returns
            \item Includes search, track, and system (cursor) targets
            \item Maybe include steerpoint (wedding cake)?
        \end{itemize}
    \end{minipage}
    }
    \caption{TWS MFD Symbology}
\end{figure}

\begin{figure}[htbp]
    \centering
    \fbox{
    \begin{minipage}[t][20mm][t]{100mm}
        \center{\large\textbf{TWS Trackfile States}}
        \begin{itemize}
            \item Show MFD symbols for search, track, system, cursor, and bugged targets
        \end{itemize}
    \end{minipage}
    }
    \caption{Trackfile MFD symbology}
\end{figure}
    
\begin{tcoloritemize}
    \blueitem{Search Target}{
    \begin{subitemize}
        \item \textbf{Initial state for radar returns}
        \begin{itemize}
            \item Not enough information to generate track
            \item Automatic transition to track after 1-2 radar sweeps
        \end{itemize}
        \item \textbf{Display} --- like RWS radar brick
    \end{subitemize}}
    \blueitem{Track Target}{
    \begin{subitemize}
        \item \textbf{FCR automatically correlates radar return data to form track files}
        \begin{itemize}
            \item Tracks target velocity, angle, altitude etc.
            \item \underline{Maximum of 10 trackfiles}, lowest priority dropped if exceeded
        \end{itemize}
        \item \textbf{Can be manually transitioned to higher modes by pilot to prioritize}
    \end{subitemize}}
    \blueitem{System Target}{
    \begin{subitemize}
        \item \textbf{Allows pilot to designate targets of interest}
        \begin{itemize}
            \item For monitoring
            \item For future weapons employment
        \end{itemize}
        \item \textbf{No additional radar energy allocated to System Targets}
    \end{subitemize}}
    \blueitem{Cursor Target}{
    \begin{subitemize}
        \item \textbf{Acquisition Cursor ``snaps'' to nearby System Targets} --- therefore named Cursor Targets
        \item \textbf{Scan centered on Cursor Target}
        \begin{itemize}
            \item 3 bar elevation
            \item $\pm$25 deg azimuth pattern 
            \item Reduces chance of losing track
        \end{itemize}
    \end{subitemize}}
    \blueitem{Bugged Target}{
    \begin{subitemize}
        \item \textbf{Selected for weapons employment}
        \begin{itemize}
            \item Can guide AIM-120C (w/o STT Lock)
            \item DLZ displayed if missile selected
            \item Launched AIM-120 will continue to guide even if another target is bugged --- \underline{allows for multi-target engagement}
        \end{itemize}
        \item \textbf{Scan Centered on Bugged Target}
        \begin{itemize}
            \item 3 bar elevation
            \item $\pm$25 deg azimuth pattern 
            \item Reduces chance of losing track
        \end{itemize}
        \item \textbf{TMS Right --- Cycles through System Targets in range order} (bugs closest if none bugged)
    \end{subitemize}}
    \blueitem{Spotlight}{
    \begin{subitemize}
        \item \textbf{Narrow scan centered on acquisition cursor}
        \begin{itemize}
            \item 4 bar elevation 
            \item \pm10 deg azimuth
        \end{itemize}
        \item \textbf{Allows pilot to override TWS scan priority}
        \item \textbf{Activated by holding TMS Forward >1 sec}
    \end{subitemize}}
\end{tcoloritemize}

\begin{figure}[htbp]
    \centering
    \fbox{
    \begin{minipage}[t][75mm][t]{100mm}
        \center{\large\textbf{MFD --- FCR --- TWS Search --- Bugged Target}}
        \begin{itemize}
            \item TWS search page with Bugged Target
            \item Shows weapon symbology and steering cues
        \end{itemize}
    \end{minipage}
    }
    \caption{TWS Bugged Target MFD Symbology}
\end{figure}

\marginfigeometry

\subsubsection{SELECT TWS MODE}
\begin{checklistenumerate}
    \blueitem{FCR Switch}{\textbf{FCR}}
    \blueitem{Desired MFD}{\textbf{FCR Page}, verify \textbf{SOI}}
    \blueitem{Radar Mode}{
        \textbf{CRM} (default), verify

        \begin{subitemize}
            \item \textbf{Dogfight/Missile Override} --- \textbf{NORM}
            \item \textbf{Radar Mode (OSB 1)} --- shows \textbf{CRM}
        \end{subitemize}
    }
    \blueitem{CRM Submode}{
        \textbf{TWS}, cycle via 

        \begin{subitemize}
            \item \textbf{TMS Right (long)} / \textbf{OSB 2} 
        \end{subitemize}
    }
\end{checklistenumerate}

\subsubsection{MULTI-TARGET ACQUISITION}
\begin{checklistenumerate}
    \blueitem{Track Target Acquisition}{
    \marginpar{
        \captionsetup{type=figure}
        \fbox{
            \begin{minipage}[t][60mm][t]{\marginparwidth}
                \center{\textbf{Search / Track Target Symbology}}
                \begin{itemize}[leftmargin=1em]
                    \item show both search and track target symbology
                    \item maybe format as a diagram/flow chart?
                \end{itemize}
            \end{minipage}
        }
        \caption{Search / Track Target Symbology}
    }
    \begin{subenumerate}
        \item Correlate onboard/offboard sensors to locate targets
        \begin{itemize}
            \item raw radar returns, RWR pings
            \item AWACS calls, datalink targets
        \end{itemize}
        \item Place targets within radar scan volume
        \item FCR automatically generates Track Targets once sufficient data available
    \end{subenumerate}}
    \blueitem{System Target Acquisition}{
    \begin{subenumerate}
        \item \textbf{Target} \dotfill under Acquisition Cursor
        \item \textbf{TMS} \dotfill \textbf{Forward} 
    \end{subenumerate}
    Repeat for all desired Track Targets,
    or to upgrade all Track Targets:
    \begin{subenumerate}
        \item \textbf{TMS} \dotfill \textbf{Right}
    \end{subenumerate}}
    \blueitem{Upgrade to Bugged Target}{
    \marginpar{
        \captionsetup{type=figure}
        \fbox{
            \begin{minipage}[t][60mm][t]{\marginparwidth}
                \center{\textbf{System / Cursor / Bugged Target Symbology}}
                \begin{itemize}[leftmargin=1em]
                    \item show system / bugged target symbology
                    \item maybe format as a diagram/flow chart?
                    \item maybe combine into one fig wth search/track target fig?
                \end{itemize}
            \end{minipage}
        }
        \caption{System / Cursor /  Bugged Target Symbology}
    }
    \begin{subenumerate}
        \item \textbf{Target} \dotfill under Acquisition Cursor
        \item \textbf{TMS} \dotfill \textbf{Forward}
    \end{subenumerate}
    To select closest System Target 
    \begin{subenumerate}
        \item \textbf{TMS} \dotfill \textbf{Right}
    \end{subenumerate}
    To cycle through System Targets in range order
    \begin{subenumerate}
        \item \textbf{TMS} \dotfill \textbf{Right}
    \end{subenumerate}}
    \blueitem{STT Lock}{(if desired)
    \begin{subenumerate}
        \item \textbf{Bugged Target} \dotfill under Acq Cursor
        \item \textbf{TMS} \dotfill \textbf{Forward}
    \end{subenumerate}}
\end{checklistenumerate}

\marginfigrestore

\subsection{ACM}
\label{subsec:acm}
\begin{tcoloritemize}
    \blueitem{ACM}{
    \textbf{A}ir \textbf{C}ombat \textbf{M}ode

    \begin{subitemize}
        \item \textbf{Used to automatically lock targets while maneuvering}
        \item \textbf{Sub-Modes}
        \begin{itemize}
            \item \textbf{HUD Scan} --- 30x20 deg 
            \item \textbf{BORE} --- Boresight (3x3 deg)
            \item \textbf{Vertical Scan} --- 10x60 deg
            \item \textbf{Slewable} --- WIP
        \end{itemize}
    \end{subitemize}}
\end{tcoloritemize}

\begin{figure}[htbp]
    \centering
    \begin{tikzpicture}[auto, node distance=10mm, x=1mm, y=1mm, ultra thick,
        % >={Stealth[length=6.75pt, width=4pt, inset=2pt]}
        >={Latex[round]}
        ]
        
        % \node[<options>](<coordinate name>)at(<coordinate>){<text>};
        \node[
            hyperref node=subsec:acm,
            diamond,
            draw,
        ](acm)at(0,0){\textbf{ACM}};
        \node[
            rectangle,
            rounded corners,
            minimum width=20mm,
            minimum height=7.5mm,
            draw,
        ](bore)at(0,20){\textbf{BORE}};
        \node[
            rectangle,
            rounded corners,
            minimum width=20mm,
            minimum height=7.5mm,
            draw,
        ](hud)at(30,0){\textbf{HUD}};
        \node[
            rectangle,
            rounded corners,
            minimum width=20mm,
            minimum height=7.5mm,
            draw,
        ](vert)at(0,-20){\textbf{Vertical}};
        \node[
            rectangle, 
            rounded corners,
            minimum width=80mm,
            minimum height=7.5mm,
            draw, 
        ](stt)at(0,40){\textbf{STT}};
                
        % Lines
        \draw [->, thick]
            (acm) -- node[pos=0.5, left]{\scriptsize\textbf{TMS FWD}} (bore);
        \draw [->, thick]
            (acm) -- node[pos=0.5, above]{\scriptsize\textbf{TMS}}node[pos=0.5, below]{\scriptsize\textbf{RIGHT}} (hud);
        \draw [->, thick]
            (acm) -- node[pos=0.5, left]{\scriptsize\textbf{TMS AFT}} (vert);
        \draw [dashed, ->, thick]
            let
                \p1=(bore.north),
                \p2=(stt.south),
            in
                (\p1) -- node[pos=0.5, left]{\scriptsize\textbf{Automatic}} (\x1,\y2);
        \draw [dashed, ->, thick]
            let
                \p1=(hud.north),
                \p2=(stt.south),
            in
                (\p1) -- node[pos=0.5, left]{\scriptsize\textbf{Automatic}} (\x1,\y2);
        \draw [dashed, rounded corners, ->, thick]
            let
                \p1=(vert.west),
                \p2=(stt.south),
            in
                (\p1) -- (\x1-20mm,\y1) -- node[pos=0.5, left]{\scriptsize\textbf{Automatic}} (\x1-20mm,\y2);
                
    \end{tikzpicture}
    \caption{\textbf{ACM Radar Modes Overview}}
    \label{fig:acmoverview}
\end{figure}

\begin{figure}[htbp]
    \centering
    \fbox{
    \begin{minipage}[t][75mm][t]{100mm}
        \center{\large\textbf{MFD --- FCR --- ACM}}
        \begin{itemize}
            \item Show default ACM page
            \item clearly mark what the ``FCR format'' is (referenced in text)
        \end{itemize}
    \end{minipage}
    }
    \caption{MFD page for FCR ACM}
\end{figure}

\begin{figure}[htbp]
    \centering
    \fbox{
    \begin{minipage}[t][40mm][t]{100mm}
        \center{\large\textbf{ACM Scan Patterns}}
        \begin{itemize}
            \item Show different scan patterns?
            \item Unsure if this figure is necessary
        \end{itemize}
    \end{minipage}
    }
    \caption{ACM Scan Patterns}
\end{figure}

\begin{tcoloritemize}
    \blueitem{HUD Submode}{
    \begin{subitemize}
        \item \textbf{30x20 deg scan} --- slightly larger than HUD
        \item \textbf{Lock Range} --- 10 nm
        \item \textbf{Selected with TMS Right} --- default mode upon ACM selection
        \item \textbf{Displays}
        \begin{itemize}
            \item \textbf{FCR Format} --- displays \textbf{ACM 20}
            \item \textbf{HUD} --- no special symbology
        \end{itemize}
    \end{subitemize}}
    \blueitem{BORE Submode}{
    \begin{subitemize}
        \item \textbf{Small, 1-beamwidth scan}
        \begin{itemize}
            \item centered 3 deg below gun cross
            \item useful for precisely locking up target
        \end{itemize}
        \item \textbf{Lock Range} --- 20 nm
        \item \textbf{Selected with TMS Forward}
        \item \textbf{Scan slaves to HMD} (if equipped and powered)
        \item \textbf{Displays}
        \begin{itemize}
            \item \textbf{FCR Format} --- displays \textbf{ACM BORE}
            \item \textbf{HUD} --- Boresight Cross at center of radar scan zone
            \item \textbf{HMD} --- Oval centered on HMD aiming cross
        \end{itemize}
    \end{subitemize}}
    \blueitem{Vertical \break Submode}{
    \begin{subitemize}
        \item \textbf{10x60 deg scan}
        \begin{itemize}
            \item centered 23 deg above gun cross
            \item useful during turning engagement to lock target ``across the circle''
        \end{itemize}
        \item \textbf{Lock Range} --- 10 nm
        \item \textbf{Selected with TMS Aft}
        \item \textbf{Displays}
        \begin{itemize}
            \item \textbf{FCR Format} --- displays \textbf{ACM 60}
            \item \textbf{HUD} --- Vertical line
        \end{itemize}
    \end{subitemize}}
    \blueitem{Slewable \break Submode}{
    \textbf{Work In Progress}

    \begin{subitemize}
        \item \textbf{Scan} --- WIP
        \item \textbf{Lock Range} --- WIP
        \item \textbf{Slew} --- \textbf{CURSOR/ENABLE Control}
        \item \textbf{Displays}
        \begin{itemize}
            \item \textbf{FCR Format} --- displays \textbf{ACM SLEW}
            \item \textbf{HUD} --- WIP
        \end{itemize}
    \end{subitemize}}
\end{tcoloritemize}

\marginfigeometry

\subsubsection{HUD / BORE / VERTICAL ACQUISITION}
\begin{checklistenumerate}
    \blueitem{FCR Setup}{
    \marginpar{
        \captionsetup{type=figure}
        \fbox{
            \begin{minipage}[t][60mm][t]{\marginparwidth}
                \center{\textbf{HUD / BORE / VERTICAL Symbology}}
                \begin{itemize}[leftmargin=1em]
                    \item maybe show the HUD symbols for the different modes?
                \end{itemize}
            \end{minipage}
        }
        \caption{HUD / BORE / VERTICAL Symbology}
    }
    \begin{subenumerate}
        \item \textbf{FCR Switch} \dotfill \textbf{FCR}
        \item \textbf{Desired MFD} \dotfill \textbf{FCR Page}
    \end{subenumerate}}
    \blueitem{Enter ACM}{
    \begin{subenumerate}
        \item \textbf{Dogfight/Missile Override} \dotfill \textbf{DGFT}
        \item \textbf{Radar Mode (OSB 2)} \dotfill verify \textbf{ACM}
    \end{subenumerate}}
    \blueitem{Select ACM Submode}{
    \begin{subitemize}
        \item \textbf{HUD} (default ACM mode) \dotfill \textbf{TMS Right}
        \item \textbf{Bore} \dotfill \textbf{TMS Forward}
        \item \textbf{Vertical} \dotfill \textbf{TMS Aft}
    \end{subitemize}}
    \blueitem{Target Acquisition}{
    \begin{subenumerate}
        \item Maneuver aircraft to place target within selected ACM scan volume 
        \item Wait for automatic transition to STT 
    \end{subenumerate}}
\end{checklistenumerate}

\subsubsection{HMD ACQUISITION}
\begin{checklistenumerate}
    \blueitem{FCR/MFD Setup}{
    \marginpar{
        \captionsetup{type=figure}
        \fbox{
            \begin{minipage}[t][60mm][t]{\marginparwidth}
                \center{\textbf{HMD Symbology}}
                \begin{itemize}[leftmargin=1em]
                    \item maybe show the HMD symbols?
                \end{itemize}
            \end{minipage}
        }
        \caption{HMD Symbology}
    }
    \begin{subenumerate}
        \item \textbf{FCR Switch} \dotfill \textbf{FCR}
        \item \textbf{Desired MFD} \dotfill \textbf{FCR Page}
        \item \textbf{HMD Brightness} \dotfill \textbf{On}
    \end{subenumerate}}
    \blueitem{Enter ACM}{
    \begin{subenumerate}
        \item \textbf{Dogfight/Missile Override} \dotfill \textbf{DGFT}
        \item \textbf{Radar Mode (OSB 2)} \dotfill verify \textbf{ACM}
    \end{subenumerate}}
    \blueitem{Select ACM Bore Submode}{
    \begin{subitemize}
        \item \textbf{Bore} \dotfill \textbf{TMS Forward}
    \end{subitemize}}
    \blueitem{Target Acquisition}{
    \begin{subenumerate}
        \item Maneuver aircraft to place target within 60 deg of nose 
        \item Place target within HMD acquisition circle
        \item Wait for automatic transition to STT 
    \end{subenumerate}}
\end{checklistenumerate}

\subsubsection{SLEWABLE ACQUISITION --- WIP}

\marginfigrestore

\subsection{STT}
\label{subsec:stt}
\begin{tcoloritemize}
    \blueitem{STT}{
    \textbf{S}ingle \textbf{T}arget \textbf{T}rack
    \begin{subitemize}
        \item \textbf{FCR continually scans one target}
        \begin{itemize}
            \item high update frequency \& precision for weapon guidance
            \item \underline{target RWR will detect STT lock}
        \end{itemize}
        \item \textbf{FCR search ceases during STT lock}
        \item \textbf{Entered by locking target from CRM (TWS/RWS) or ACM}
    \end{subitemize}}
    \blueitem{Display}{
    \begin{subitemize}
        \item \textbf{Target state} shown at top of FCR page
        \begin{itemize}
            \item Aspect Angle 
            \item Ground Track 
            \item Airspeed 
            \item Closure Rate
        \end{itemize}
    \end{subitemize}}
    \blueitem{NCTR}{
    \textbf{N}on-\textbf{C}ooperative \textbf{T}arget \textbf{R}ecognition
    \begin{subitemize}
        \item \textbf{FCR attempts to identify locked target}
        \begin{itemize}
            \item measures turbine parameters to produce \underline{likely} contact aircraft type
            \item NCTR is only available in STT
        \end{itemize}
        \item \textbf{NCTR imposes additional requirements}
        \begin{itemize}
            \item target must be within 20-25nm
            \item Radar must ``see'' compressor/turbine blades
        \end{itemize}
        \item \textbf{Activated with IFF interrogation (TMS Left long)}
        \item \textbf{Hostile NCTR identification counts towards ROE matrix} (combined with IFF return)
    \end{subitemize}}
\end{tcoloritemize}

\begin{figure}[htbp]
    \centering
    \fbox{
    \begin{minipage}[t][75mm][t]{100mm}
        \center{\large\textbf{MFD --- FCR --- STT}}
        \begin{itemize}
            \item Show standard STT lock
            \item should clearly mark target state, steering cues, dlz, NCTR
        \end{itemize}
    \end{minipage}
    }
    \caption{MFD page during STT lock}
\end{figure}

\clearpage 

\subsection{IFF}

\begin{tcoloritemize}
    \blueitem{IFF}{
    \textbf{I}dentify \textbf{F}riend or \textbf{F}oe
    \begin{subitemize}
        \item \textbf{Unknown contacts can be ``interrogated'' to determine if friendly}
        \item \textbf{Works via transponder system}
        \begin{itemize}
            \item Interrogater sends coded pulse
            \item Friendly transponder return matching coded pulse
        \end{itemize}
        \item \textbf{Controlled via dedicated IFF panel}
        \begin{itemize}
            \item Must be powered on separately from FCR to function
        \end{itemize}
        \item \textbf{Friendly IFF replies displayed as green circles on FCR MFD}
    \end{subitemize}}
    \blueitem{Manual \break Interrogation}{
    Manual IFF interrogation can be initiated in 2 modes

    \begin{subitemize}
        \item \textbf{Scan} --- Interrogates entire radar scan volume
        \begin{itemize}
            \item Activated by pressing \textbf{TMS Left (short)}
            \item \textbf{SCAN} appears next to \textbf{OSB 16} on FCR page
        \end{itemize}
        \item \textbf{LOS} (\textbf{L}ine \textbf{O}f \textbf{S}ight) --- Interrogates locked target or scan volume around acquisition cursor
        \begin{itemize}
            \item Actived by pressing \textbf{TMS Left (long)}
            \item \textbf{LOS} appears next to \textbf{OSB 16} on FCR page
        \end{itemize}
    \end{subitemize}}
\end{tcoloritemize}

\warningbox{
    \textbf{Lack of IFF return does \underline{not} necessarily mean target is hostile} 
    \begin{itemize}
        \item system relies on active participation of interrogated target
        \item depending on ROE, additional information may be required prior to weapon employment
    \end{itemize}
}

\notebox{
    \textbf{IFF has independent transponder, can function even when FCR is not radiating}
}

% \subsubsection{IFF DED PAGE}

% \begin{figure}[htbp]
%     \centering
%     \fbox{
%     \begin{minipage}[t][30mm][t]{100mm}
%         \center{\large\textbf{IFF DED Display}}
%         \begin{itemize}
%             \item Standard DED display (with IFF information bottom left)
%         \end{itemize}
%     \end{minipage}
%     }
%     \caption{DED IFF Status}
% \end{figure}

% \subsubsection{INTG DED PAGE}

% \subsubsection{IFF CONTROL PANEL}

% \begin{figure}[htbp]
%     \centering
%     \fbox{
%     \begin{minipage}[t][75mm][t]{100mm}
%         \center{\large\textbf{IFF Panel}}
%         \begin{itemize}
%             \item Show IFF panel (\underline{just} IFF panel)
%         \end{itemize}
%     \end{minipage}
%     }
%     \caption{IFF Control Panel}
% \end{figure}


% \begin{tcoloritemize}
%     \blueitem{Master Switch}{
%     Controls operating mode of IFF system when \textbf{C\&I Switch} is set to \textbf{BACKUP}

%     \begin{subitemize}
%         \item \textbf{OFF} --- removes power from system
%         \item \textbf{STBY} --- system is powered but in standby mode
%         \item \textbf{NORM} --- normal operating mode
%     \end{subitemize}
%     }
%     \blueitem{M-4 Code Switch}{}
%     \blueitem{C \& I Switch}{
%     Selects source for \textbf{C}ommunications \textbf{\&} \textbf{I}FF control

%     \begin{subitemize}
%         \item \textbf{BACKUP}
%         \begin{itemize}
%             \item \textbf{UHF} and \textbf{IFF} panels used to control relevant systems
%             \item Can be activated without system fault
%         \end{itemize}
%         \item \textbf{UFC} --- \textbf{U}p \textbf{F}ront \textbf{C}ontrols
%         \begin{itemize}
%             \item Sets input source as UFC (ICP \& DED) for both IFF and UHF
%         \end{itemize}
%     \end{subitemize}}
%     \blueitem{Enable Switch}{}
%     \blueitem{Mode 1 / 3 Selectors}{}
%     \blueitem{Mode 4 Reply Switch}{}
%     \blueitem{Mode 4 Monitor Switch}{}
% \end{tcoloritemize}

\marginfigeometry

\subsubsection{IFF INTERROGATION}
\begin{checklistenumerate}
    \blueitem{Prerequisites}{
    \marginpar{
        \captionsetup{type=figure}
        \fbox{
            \begin{minipage}[t][40mm][t]{\marginparwidth}
                \center{\textbf{IFF Control Panel Location}}
                \begin{itemize}[leftmargin=1em]
                    \item show panel location like in start up
                \end{itemize}
            \end{minipage}
        }
        \caption{IFF Control Panel}
    }
    \begin{subitemize}
        \item \textbf{C\&I Switch} \dotfill \textbf{UFC}
        \item \textbf{IFF Master Mode Switch} \dotfill \textbf{NORM}
        \item \textbf{Desired MFD} \dotfill \textbf{FCR Page}, \textbf{SOI}
    \end{subitemize}
    
    Not necessary, but highly recommended
    
    \begin{subitemize}
        \item \textbf{FCR Switch} \dotfill \textbf{FCR}
    \end{subitemize}}
    \blueitem{Locate Targets}{
    \begin{subenumerate}
        \item Correlate onboard/offboard sensors
        \begin{itemize}
            \item raw radar returns, RWR pings
            \item AWACS calls, datalink targets
        \end{itemize}
        \item Place target within radar scan volume
    \end{subenumerate}}
    \blueitem{Interrogate}{
    \marginpar{
        \captionsetup{type=figure}
        \fbox{
            \begin{minipage}[t][80mm][t]{\marginparwidth}
                \center{\textbf{IFF Return Symbology}}
                \begin{itemize}[leftmargin=1em]
                    \item show symbology elements for IFF returns (friendly and non friendly)
                \end{itemize}
            \end{minipage}
        }
        \caption{IFF Symbology}
    }

    \smallskip
    To interrogate the entire radar scan volume

    \begin{subenumerate}
        \item \textbf{TMS} \dotfill \textbf{Left (short)}
    \end{subenumerate}

    To interrogate the currently locked target or volume around the cursor 

    \begin{subenumerate}
        \item \textbf{TMS} \dotfill \textbf{Left (long)}
    \end{subenumerate}}
    \blueitem{Evaluate Returns}{After 1-3 sec IFF replies should appear

    \begin{subitemize}
        \item Friendlies are marked by green circles
        \item Depending on avionics and radar mode, radar tracks may be marked in green to indicate friendly
    \end{subitemize}}
\end{checklistenumerate}

\marginfigrestore

\clearpage 

\section{AIR-TO-GROUND MODES --- WIP}
\label{sec:fcr-ag}

% \subsection{GM}

% \subsection{GMT}

\cleardoublepage