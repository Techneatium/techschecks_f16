\section{INTRODUCTION}
\subsection{HOW TO WIN}
\begin{tcoloritemize}
    \blueitem[Goal]
    The ultimate goal of any air-to-air engagement is to kill the enemy without being killed. 

    \medskip
    \textbf{Desired end-state:}
    \begin{enumerate}[label=\bfseries(\arabic*)]
        \item bandit destroyed by fighter ordnance
        \label{subsec:ttp_aa:intro:howto:endstate:bandit}
        \item fighter undamaged, remains combat effective
        \label{subsec:ttp_aa:intro:howto:endstate:fighter}
    \end{enumerate}

    \blueitem[BVR Killchain]
    We work backwards from the desired end-state 
    to missile launch to define the killchain:
    \begin{enumerate}[label=\textbf{(\alph*)}]
        \item missile fuzes, bandit destroyed
        \label{subsec:ttp_aa:intro:howto:killchain:fuzes}
        \item terminal guidance to intercept
        \label{subsec:ttp_aa:intro:howto:killchain:terminal}
        \item missile seeker acquisition
        \label{subsec:ttp_aa:intro:howto:killchain:acquired}
        \item mid-course guidance to seeker basket
        \label{subsec:ttp_aa:intro:howto:killchain:midcourse}
        \item missile launch
        \label{subsec:ttp_aa:intro:howto:killchain:launch}
    \end{enumerate}

    \blueitem[Probability of Kill]
    Some steps in the killchain are probabalistic 
    \ref{subsec:ttp_aa:intro:howto:killchain:fuzes},
    \ref{subsec:ttp_aa:intro:howto:killchain:acquired}.
    However, as it is impossible to perfectly predict bandit actions, 
    we treat 
    \ref{subsec:ttp_aa:intro:howto:killchain:terminal},
    \ref{subsec:ttp_aa:intro:howto:killchain:midcourse}
    as probabalistic.

    \medskip
    The probability of kill P\textsubscript{K} is the combined chance of all events in the killchain occuring successfully.

    \blueitem[Devising Tactics]
    To achieve \ref{subsec:ttp_aa:intro:howto:endstate:bandit},
    we maximize P\textsubscript{K} through the figher-effected links in the killchain
    \ref{subsec:ttp_aa:intro:howto:killchain:midcourse},
    \ref{subsec:ttp_aa:intro:howto:killchain:launch}.

    \medskip
    Simultaneously, we must ensure that our tactics minimize bandit P\textsubscript{K}, 
    ensuring \ref{subsec:ttp_aa:intro:howto:endstate:fighter}
\end{tcoloritemize}

\clearpage

\subsection{OVERVIEW}
\begin{tcoloritemize}
    \blueitem[Fundamentals] 
    In \cref{sec:ttp_aa:bvr_fundamentals}, 
    we examine the fundamentals of BVR engagements, 
    explaining the typical flight profile of an active radar missile,
    define the ranges which are relevant for tactical employment,
    as well as how the fighter and bandit can each effect these ranges

    \blueitem[Intercept \break Timelines]
    In \cref{sec:ttp_aa:timelines}, 
    we then discuss basic intercept timelines, 
    defining how to prosecute individual contacts while leveraging the capabilities of the AIM-120

    \blueitem[Element Tactics]

    \blueitem[Four-Ship \break Tactics]
\end{tcoloritemize}

\clearpage