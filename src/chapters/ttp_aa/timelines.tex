\section{INTERCEPT TIMELINES}
\label{sec:ttp_aa:timelines}

\subsection[AR FLOW]{ACTIVE-RADAR MISSILE FLOW}

\begin{tcoloritemize}
    \blueitem[Launch-and-Leave]
    Launch-and-leave tactics utilize AIM-120 terminal guidance independence 
    to defeat bandit missiles though out maneuvers, 
    with the fighter turning cold once AIM-120 has gone active.

    \bigskip
    \textbf{Skate} \hfill \textbf{see \cref{subsec:ttp_aa:timeline:skate}}\\
    AIM-120 launched before a briefed tranisition range 
    such that it goes active before fighters reach a desired out range.
    Fighters go out before recommitting for a second launch.
    
    \bigskip
    \textbf{Short-Skate} \hfill \textbf{see \cref{subsec:ttp_aa:timeline:shortskate}}\\
    AIM-120 launched such that fighter can go out before reaching MAR.

    \bigskip
    Launch-and-leave tactics preserve a kinematic defense by going out

    \blueitem[Launch-and-Decide]
    Launch-and-decide tactics also utilize AIM-120 terminal guidance independence, 
    allowing fighter to decide whether to abort out or continue in.

    \bigskip
    \textbf{Banzai} \hfill \textbf{see \cref{subsec:ttp_aa:timeline:banzai}}\\
    Fighter launches before a briefed decision range and goes into notch for predetermined duration. 
    Can abort out or continue in depending on if spiked/naked and bandit maneuver.

    Typically employed with elements cranking in opposite directions, 
    increasing chance of 1 fighter being naked.

    \bigskip
    Launch-and-decide tactics do NOT necessarily preserve a kinematic defense, 
    but allow fighters to decide to press the attack.
\end{tcoloritemize}


\subsection{RANGE DEFINITIONS}

\begin{tcoloritemize}
    \blueitem[Bandit WEZ] \textbf{W}eapon \textbf{E}ngagement \textbf{Z}one

    \medskip
    range at which bandit weapons can engage fighter, synonomous with bandit R\textsubscript{F-pole} in this document

    \blueitem[MAR] \textbf{M}inimum \textbf{A}bort \textbf{R}ange

    \medskip
    minimum range at which fighter can perform an abort maneuver to kinematically defeat any launched bandit weapons 

    \medskip
    \textbf{MAR} = max bandit R\textsubscript{F-pole} + fighter turn radius
    \blueitem[MSR] \textbf{M}inimum \textbf{S}hot \textbf{R}ange

    \medskip
    minimum range at which AIM-120 will go active before fighter reaches MAR if launched

    \blueitem[DOR] \textbf{D}esired \textbf{O}ut \textbf{R}ange

    \medskip
    minimum range at which fighter can go out, defeating bandit missiles, 
    before recommitting to a second employment with launch-and-decide tactics

    \blueitem[TR] \textbf{T}ransition \textbf{R}ange

    \medskip
    minimum range at which AIM-120 will go active before fighter reaches DOR if launched
    \blueitem[MTR] \textbf{M}inimum \textbf{T}argeting \textbf{R}ange

    \medskip
    minimum range for flight to begin targeting and execute launch-and-leave tactics
    \blueitem[MRR] \textbf{M}inimum \textbf{R}ecommit \textbf{R}ange

    \medskip
    minimum range at which a fighter which is out can recommit, 
    retarget and employ an AIM-120 before going out again at MAR

    \blueitem[DR] \textbf{D}ecision \textbf{R}ange

    \medskip
    minimum range at which fighter can execute briefed notch maneuver for launch-and-decide tactics

    \medskip

    \textbf{DR} = max bandit R\textsubscript{F-pole} + bandit closure \times \ t\textsubscript{notch}
\end{tcoloritemize}

\notebox{%
    \small%
    \textbf{It is important to note that:}
    \begin{itemize}
        \item The timelines described in \cref{subsec:ttp_aa:timeline:banzai,subsec:ttp_aa:timeline:skate,subsec:ttp_aa:timeline:shortskate}
        are to help illustrate core concepts of intercept timelines, 
        \textbf{NOT} to be followed as absolute procedures.
        \item To derive a true timeline, you must define your assumptions.
        \begin{itemize}
            \item what is the bandit state (speed/altitude)?
            \item what is the performance of bandit missiles?
            \item what tactics will bandit most likely employ?
        \end{itemize}
        \item Specifically, defining the bandit WEZ / R\textsubscript{f-pole}, 
        which determine the MAR, 
        relies on accurate models of of threat missiles, aircraft, and tactics.
        \item \Cref{fig:ttp_aa:timeline:banzai,fig:ttp_aa:timeline:skate,fig:ttp_aa:timeline:shortskate} 
        are not to scale, 
        any crank maneuvers are ommitted for compactness and visual clarity.
        \item Flight members should prioritize tasks during an intercept to maintain SA.
        Typical order of priority is
        \begin{enumerate}
            \item formation
            \item sensors
            \item communications
        \end{enumerate}
    \end{itemize}
}

\marginfigeometry

\subsection{BANZAI TIMELINE}
\label{subsec:ttp_aa:timeline:banzai}

\begin{checklistenumerate}[start=0]
    \blueitem[Pre-commit] maintain SA
    
    \begin{itemize}
        \item \textbf{Comms} --- monitor AWACS
        \item \textbf{Sensors} --- sanitize airspace
    \end{itemize}

    \blueitem[Commit]%
    \label{subsec:ttp_aa:timeline:banzai:commit}
    \marginpar{
        \captionsetup{type=figure}
        \centering
        \begin{tikzpicture}[figstyle]

            % coordinates
            \coordinate (fighter_start) at (0,0);
            \coordinate (bandit) at (5,75);

            \coordinate (cr) at (0,0);
            \coordinate (mtr) at (0,10);
            \coordinate (sort) at (0,17.5);
            \coordinate (shoot) at (0,25);
            \coordinate (dr) at (0,35);
            \coordinate (fighter_banzai) at (20,55);
            \coordinate (fighter_abort) at (20,30);
            \coordinate (mar) at (5,50);
            \coordinate (wez) at (0,60);

            % range lines
            \draw[thin]
                (25,0) -- (25,75);

            \path let \p1=(bandit) in 
            node[font=\footnotesize,anchor=west] at (25,\y1) {BANDIT};
            \path let \p1=(wez) in 
            node[font=\footnotesize,anchor=west] at (25,\y1) {WEZ};
            \path let \p1=(mar) in 
            node[font=\footnotesize,anchor=west] at (25,\y1) {MAR};
            \path let \p1=(dr) in 
            node[font=\footnotesize,anchor=west] at (25,\y1) {DR};
            \path let \p1=(mtr) in 
            node[font=\footnotesize,anchor=west] at (25,\y1) {MTR};
            \path let \p1=(cr) in 
            node[font=\footnotesize,anchor=west] at (25,\y1) {CR};

            \draw[thin, dashed] let \p1=(wez) in  
                (25,\y1) -- ++(-30, 0);
            \draw[thin, dashed] let \p1=(mar) in  
                (25,\y1) -- ++(-30, 0);
            \draw[thin, dashed] let \p1=(dr) in  
                (25,\y1) -- ++(-25, 0);
            \draw[thin, dashed] let \p1=(mtr) in  
                (25,\y1) -- ++(-25, 0);
            \draw[thin, dashed] let \p1=(cr) in  
                (25,\y1) -- ++(-25, 0);

            % bandit wez
            \draw[fill=red!40]
                (bandit)
                -- ++(-60:15)
                arc (-60:-120:15)
                -- (bandit);
            
            % timeline
            \draw[->] 
                (fighter_start) -- 
                node[below, pos=0]{
                    \includegraphics[
                    width=7.5mm,
                ]{diagrams/aircraft/silhouette_f16_top.pdf}} 
                (mtr);
            \draw[->]
                (mtr)
                -- (shoot);
            \draw[->]
                (shoot)
                -- (dr);
            \draw[->, dashed]
                (dr)
                arc (180:90:5) 
                -- ++(10,0)
                arc (90:0:5) 
                -- (fighter_abort)
                node[below, pos=1, ]{
                    \includegraphics[
                        angle=180,
                        width=7.5mm,
                ]{diagrams/aircraft/silhouette_f16_top.pdf}};
            \draw[->]
                (dr)
                arc (180:90:5) 
                -- ++(10,0)
                arc (-90:0:5) 
                -- (fighter_banzai)
                node[above, pos=1, ]{
                    \includegraphics[
                        angle=0,
                        width=7.5mm,
                ]{diagrams/aircraft/silhouette_f16_top.pdf}};

            % bandit
            \node[] at (bandit) {
                \includegraphics[
                    angle=180,
                    width=7.5mm,
            ]{diagrams/aircraft/silhouette_f16_top.pdf}};

            % labels
            \node[left, align=right, font=\small] at (cr) {
                \ref{subsec:ttp_aa:timeline:banzai:commit}
            };
            \node[left, align=right, font=\small] at (mtr) {
                \ref{subsec:ttp_aa:timeline:banzai:target}
            };
            \node[left, align=right, font=\small] at (sort) {
                \ref{subsec:ttp_aa:timeline:banzai:sort}
            };
            \node[left, align=right, font=\small] at (shoot) {
                \ref{subsec:ttp_aa:timeline:skate:shoot}
            };
            \node[left, align=right, font=\small] at (dr) {
                \ref{subsec:ttp_aa:timeline:banzai:notch}
            };
            \node[left, align=right, font=\small] at (fighter_banzai) {
                \ref{subsec:ttp_aa:timeline:banzai:banzai}
            };
            \node[left, align=right, font=\small] at (fighter_abort) {
                \ref{subsec:ttp_aa:timeline:banzai:abort}
            };

        \end{tikzpicture}
        \caption{Banzai timeline}
        \label{fig:ttp_aa:timeline:banzai}
    }%
    \textbf{--- no later than CR}
    \begin{itemize}
        \item AWACS picture or own FCR contacts meet briefed commit criteria
        \item flight leaves assigned patrol area
        \item start of intercept timeline
    \end{itemize}

    \blueitem[Target] \textbf{--- no later than MTR}
    \label{subsec:ttp_aa:timeline:banzai:target}
    \begin{itemize}
        \item target call indicates responsibility to \\
        engage group in accordance with ROE
        \item flight members obtain radar contact
    \end{itemize}

    \blueitem[Sort]
    \label{subsec:ttp_aa:timeline:banzai:sort}
    \begin{itemize}
        \item flight members sort contacts
        \item flight members obtain FCR lock on assigned contact
    \end{itemize}

    \blueitem[MRM Employment]
    \label{subsec:ttp_aa:timeline:banzai:shoot}
    \begin{itemize} 
        \item verify clear avenue of fire
        \item crank post launch to minimize closure
        \item \textbf{such that missile active before fighter reaches DR}
    \end{itemize}
    
    \blueitem[Notch] \textbf{--- no later than DR}
    \label{subsec:ttp_aa:timeline:banzai:notch}
    \begin{itemize} 
        \item notch predetermined time (15s)
    \end{itemize}
    \blueitem[Banzai / Abort]
    \begin{enumerate}[label=\textbf{\arabic{enumi}\alph*.}]
        \item \blue{Abort} (spiked) --- \textbf{5G slicing turn}%
        \label{subsec:ttp_aa:timeline:banzai:abort}%
        \item \blue{Banzai} (naked) --- recommit and engage%
        \label{subsec:ttp_aa:timeline:banzai:banzai}%
    \end{enumerate}
    
\end{checklistenumerate}

\clearpage

\subsection{SKATE TIMELINE}
\label{subsec:ttp_aa:timeline:skate}

\begin{checklistenumerate}[start=0]
    \blueitem[Pre-commit] maintain SA
    
    \begin{itemize}
        \item \textbf{Comms} --- monitor AWACS
        \item \textbf{Sensors} --- sanitize airspace
    \end{itemize}

    \blueitem[Commit]%
    \label{subsec:ttp_aa:timeline:skate:commit}
    \marginpar{
        \captionsetup{type=figure}
        \centering
        \begin{tikzpicture}[figstyle]

            % coordinates
            \coordinate (fighter_start) at (0,0);
            \coordinate (bandit) at (5,80);

            \coordinate (cr) at (0,0);
            \coordinate (mtr) at (0,10);
            \coordinate (sort) at (0,20);
            \coordinate (tr) at (0,30);
            \coordinate (dor) at (0,40);
            \coordinate (mrr) at (15,20);
            \coordinate (shoot2) at (5,50);
            \coordinate (mar) at (5,55);
            \coordinate (wez) at (0,65);
            \coordinate (fighter_end) at (20,50);

            % range lines
            \draw[thin]
                (25,0) -- (25,80);

            \path let \p1=(bandit) in 
            node[font=\footnotesize,anchor=west] at (25,\y1) {BANDIT};
            \path let \p1=(wez) in 
            node[font=\footnotesize,anchor=west] at (25,\y1) {WEZ};
            \path let \p1=(mar) in 
            node[font=\footnotesize,anchor=west] at (25,\y1) {MAR};
            \path let \p1=(dor) in 
            node[font=\footnotesize,anchor=west] at (25,\y1) {DOR};
            \path let \p1=(mrr) in 
            node[font=\footnotesize,anchor=west] at (25,\y1) {MRR};
            \path let \p1=(tr) in 
            node[font=\footnotesize,anchor=west] at (25,\y1) {TR};
            \path let \p1=(mtr) in 
            node[font=\footnotesize,anchor=west] at (25,\y1) {MTR};
            \path let \p1=(cr) in 
            node[font=\footnotesize,anchor=west] at (25,\y1) {CR};

            
            
            \draw[thin, dashed] let \p1=(wez) in  
                (25,\y1) -- ++(-30, 0);
            \draw[thin, dashed] let \p1=(mar) in  
                (25,\y1) -- ++(-20, 0);
            \draw[thin, dashed] let \p1=(dor) in  
                (25,\y1) -- ++(-25, 0);
            \draw[thin, dashed] let \p1=(tr) in  
                (25,\y1) -- ++(-25, 0);
            \draw[thin, dashed] let \p1=(mrr) in  
                (25,\y1) -- ++(-10, 0);
            \draw[thin, dashed] let \p1=(mtr) in  
                (25,\y1) -- ++(-25, 0);
            \draw[thin, dashed] let \p1=(cr) in  
                (25,\y1) -- ++(-25, 0);

            % bandit wez
            \draw[fill=red!40]
                (bandit)
                -- ++(-60:15)
                arc (-60:-120:15)
                -- (bandit);
            
            % timeline
            \draw[->] 
                (fighter_start) -- 
                node[below, pos=0]{
                    \includegraphics[
                    width=7.5mm,
                ]{diagrams/aircraft/silhouette_f16_top.pdf}} 
                (mtr);
            \draw[->]
                (mtr)
                -- (tr);
            \draw[->]
                (tr)
                -- (dor);
            \draw[->]
                (dor)
                arc (180:90:5) 
                -- ++(5,0)
                arc (90:0:5) 
                -- (mrr);
            \draw[->]
                (mrr)
                arc (0:-180:5) 
                -- (mar);
            \draw[->]
                (mar)
                arc (180:90:5) 
                -- ++(5,0)
                arc (90:0:5) 
                -- (fighter_end)
                node[below, pos=1, ]{
                    \includegraphics[
                        angle=180,
                        width=7.5mm,
                ]{diagrams/aircraft/silhouette_f16_top.pdf}};
            \draw[->, dashed]
                (mar)
                arc (180:90:5) 
                -- ++(5,0)
                arc (-90:0:5) 
                -- ++(0, 5)
                node[above, pos=1, ]{
                    \includegraphics[
                        angle=0,
                        width=7.5mm,
                ]{diagrams/aircraft/silhouette_f16_top.pdf}};

            % bandit
            \node[] at (bandit) {
                \includegraphics[
                    angle=180,
                    width=7.5mm,
            ]{diagrams/aircraft/silhouette_f16_top.pdf}};

            % labels
            \node[left, align=right, font=\small] at (cr) {
                \ref{subsec:ttp_aa:timeline:skate:commit}
            };
            \node[left, align=right, font=\small] at (mtr) {
                \ref{subsec:ttp_aa:timeline:skate:target}
            };
            \node[left, align=right, font=\small] at (sort) {
                \ref{subsec:ttp_aa:timeline:skate:sort}
            };
            \node[left, align=right, font=\small] at (tr) {
                \ref{subsec:ttp_aa:timeline:skate:shoot}
            };
            \node[left, align=right, font=\small] at (dor) {
                \ref{subsec:ttp_aa:timeline:skate:out}
            };
            \node[left, align=right, font=\small] at (mrr) {
                \ref{subsec:ttp_aa:timeline:skate:recommit}
            };
            \node[left, align=right, font=\small] at (shoot2) {
                \ref{subsec:ttp_aa:timeline:skate:shoot2}
            };
            \node[left, align=right, font=\small] at (mar) {
                \ref{subsec:ttp_aa:timeline:skate:abort}
            };

        \end{tikzpicture}
        \caption{Skate timeline}
        \label{fig:ttp_aa:timeline:skate}
    }%
    \textbf{--- no later than CR}
    \begin{itemize}
        \item AWACS picture or own FCR contacts meet briefed commit criteria
        \item flight leaves assigned patrol area
        \item start of intercept timeline
    \end{itemize}

    \blueitem[Target] \textbf{--- no later than MTR}
    \label{subsec:ttp_aa:timeline:skate:target}
    \begin{itemize}
        \item target call indicates responsibility to \\
        engage group in accordance with ROE
        \item flight members obtain radar contact
    \end{itemize}

    \blueitem[Sort]
    \label{subsec:ttp_aa:timeline:skate:sort}
    \begin{itemize}
        \item flight members sort contacts
        \item flight members obtain FCR lock on assigned contact
    \end{itemize}

    \blueitem[MRM Employment] \textbf{--- no later than TR}
    \label{subsec:ttp_aa:timeline:skate:shoot}
    \begin{itemize} 
        \item verify clear avenue of fire
        \item DLZ --- R\textsubscript{PI} to R\textsubscript{OPT}
        \hfill (see \cref{fig:ttp_aa:timeline:skate:dlz})\\
        manual loft to maximize performance
        \item crank post launch to minimize closure
    \end{itemize}
    
    \marginpar{
        \captionsetup{type=figure}
        \centering
        \begin{tikzpicture}[figstyle]
            \node[boxedmarfigstyle] (fig) at (0,0) {
                \includegraphics[
                    scale=0.5,
                ]{mfd/fcr_aa/aim120_subfig_dlz_prelaunch.pdf}
            };
        \end{tikzpicture}
        \caption{DLZ for skate}
        \label{fig:ttp_aa:timeline:skate:dlz}
    }

    \blueitem[Out] \textbf{--- no later than DOR, after pitbull}
    \label{subsec:ttp_aa:timeline:skate:out}
    \begin{itemize}
        \item 5G slicing turn
    \end{itemize}
    \blueitem[Recommit] (if necessary)
    \label{subsec:ttp_aa:timeline:skate:recommit}
    \begin{itemize}
        \item \textbf{range must be greater than MRR}
        \item fighter must reacquire \& sort bandit
    \end{itemize}
    \blueitem[MRM Employment] \textbf{--- no later than MAR}
    \begin{itemize}
        \item typically use launch-and-decide tactics
    \end{itemize}
    \label{subsec:ttp_aa:timeline:skate:shoot2}
    \blueitem[Abort]%
    \label{subsec:ttp_aa:timeline:skate:abort}
    \textbf{--- 5G slicing turn at MAR}
\end{checklistenumerate}

\clearpage

\subsection{SHORT SKATE TIMELINE}
\label{subsec:ttp_aa:timeline:shortskate}

\begin{checklistenumerate}[start=0]
    \blueitem[Pre-commit] maintain SA
    
    \begin{itemize}
        \item \textbf{Comms} --- monitor AWACS
        \item \textbf{Sensors} --- sanitize airspace
    \end{itemize}

    \blueitem[Commit]%
    \label{subsec:ttp_aa:timeline:shortskate:commit}
    \marginpar{
        \captionsetup{type=figure}
        \centering
        \begin{tikzpicture}[figstyle]

            % coordinates
            \coordinate (fighter_start) at (0,0);
            \coordinate (bandit) at (0,80);

            \coordinate (cr) at (0,0);
            \coordinate (mtr) at (0,10);
            \coordinate (sort) at (0,20);
            \coordinate (tr) at (0,30);
            \coordinate (msr) at (0,40);
            \coordinate (mar) at (0,55);
            \coordinate (wez) at (0,65);
            \coordinate (fighter_end) at (15,50);

            % range lines
            \draw[thin]
                (20,0) -- (20,80);

            \path let \p1=(bandit) in 
            node[font=\footnotesize,anchor=west] at (20,\y1) {BANDIT};
            \path let \p1=(wez) in 
            node[font=\footnotesize,anchor=west] at (20,\y1) {WEZ};
            \path let \p1=(mar) in 
            node[font=\footnotesize,anchor=west] at (20,\y1) {MAR};
            \path let \p1=(msr) in 
            node[font=\footnotesize,anchor=west] at (20,\y1) {MSR};
            \path let \p1=(tr) in 
            node[font=\footnotesize,anchor=west] at (20,\y1) {TR};
            \path let \p1=(mtr) in 
            node[font=\footnotesize,anchor=west] at (20,\y1) {MTR};
            \path let \p1=(cr) in 
            node[font=\footnotesize,anchor=west] at (20,\y1) {CR};

            
            
            \draw[thin, dashed] let \p1=(wez) in  
                (20,\y1) -- ++(-30, 0);
            \draw[thin, dashed] let \p1=(mar) in  
                (20,\y1) -- ++(-20, 0);
            \draw[thin, dashed] let \p1=(msr) in  
                (20,\y1) -- ++(-20, 0);
            \draw[thin, dashed] let \p1=(tr) in  
                (20,\y1) -- ++(-20, 0);
            \draw[thin, dashed] let \p1=(mtr) in  
                (20,\y1) -- ++(-20, 0);
            \draw[thin, dashed] let \p1=(cr) in  
                (20,\y1) -- ++(-20, 0);

            % bandit wez
            \draw[fill=red!40]
                (bandit)
                -- ++(-60:15)
                arc (-60:-120:15)
                -- (bandit);
            
            % timeline
            \draw[->] 
                (fighter_start) -- 
                node[below, pos=0]{
                    \includegraphics[
                    width=7.5mm,
                ]{diagrams/aircraft/silhouette_f16_top.pdf}} 
                (mtr);
            \draw[->]
                (mtr)
                -- (msr);
            \draw[->]
                (msr)
                -- (mar);
            \draw[->]
                (mar)
                arc (180:90:5) 
                -- ++(5,0)
                arc (90:0:5) 
                -- (fighter_end)
                node[below, pos=1, ]{
                    \includegraphics[
                        angle=180,
                        width=7.5mm,
                ]{diagrams/aircraft/silhouette_f16_top.pdf}};
            \draw[->, dashed]
                (mar)
                arc (180:90:5) 
                -- ++(5,0)
                arc (-90:0:5) 
                -- ++(0, 5)
                node[above, pos=1, ]{
                    \includegraphics[
                        angle=0,
                        width=7.5mm,
                ]{diagrams/aircraft/silhouette_f16_top.pdf}};

            % bandit
            \node[] at (bandit) {
                \includegraphics[
                    angle=180,
                    width=7.5mm,
            ]{diagrams/aircraft/silhouette_f16_top.pdf}};

            % labels
            \node[left, align=right, font=\small] at (cr) {
                \ref{subsec:ttp_aa:timeline:shortskate:commit}
            };
            \node[left, align=right, font=\small] at (mtr) {
                \ref{subsec:ttp_aa:timeline:shortskate:target}
            };
            \node[left, align=right, font=\small] at (sort) {
                \ref{subsec:ttp_aa:timeline:shortskate:sort}
            };
            \node[left, align=right, font=\small] at (msr) {
                \ref{subsec:ttp_aa:timeline:shortskate:shoot}
            };
            \node[left, align=right, font=\small] at (mar) {
                \ref{subsec:ttp_aa:timeline:shortskate:abort}
            };

        \end{tikzpicture}
        \caption{Short skate timeline}
        \label{fig:ttp_aa:timeline:shortskate}
    }%
    \textbf{--- no later than CR}
    \begin{itemize}
        \item AWACS picture or own FCR contacts meet briefed commit criteria
        \item flight leaves assigned patrol area
        \item start of intercept timeline
    \end{itemize}

    \blueitem[Target] \textbf{--- no later than MTR}
    \label{subsec:ttp_aa:timeline:shortskate:target}
    \begin{itemize}
        \item target call indicates responsibility to \\
        engage group in accordance with ROE
        \item flight members obtain radar contact
    \end{itemize}

    \blueitem[Sort]
    \label{subsec:ttp_aa:timeline:shortskate:sort}
    \begin{itemize}
        \item flight members sort contacts
        \item flight members obtain FCR lock on assigned contact
    \end{itemize}

    \blueitem[MRM Employment] \textbf{--- no later than MSR}
    \label{subsec:ttp_aa:timeline:shortskate:shoot}
    \begin{itemize} 
        \item verify clear avenue of fire
        \item DLZ --- R\textsubscript{TR} to R\textsubscript{PI} 
        \hfill (see \cref{fig:ttp_aa:timeline:shortskate:dlz})\\
        manual loft to maximize performance
        \item crank post launch to minimize closure
        \item launch from within TR to increase P\textsubscript{K}
    \end{itemize}
    
    \marginpar{
        \captionsetup{type=figure}
        \centering
        \begin{tikzpicture}[figstyle]
            \node[boxedmarfigstyle] (fig) at (0,0) {
                \includegraphics[
                    scale=0.5,
                ]{mfd/fcr_aa/aim120_subfig_dlz_prelaunch.pdf}
            };
        \end{tikzpicture}
        \caption{DLZ for short skate}
        \label{fig:ttp_aa:timeline:shortskate:dlz}
    }
    \blueitem[Abort]%
    \label{subsec:ttp_aa:timeline:shortskate:abort}
    \textbf{--- 5G slicing turn at MAR}
\end{checklistenumerate}

\notebox{%
    \small\textbf{%
        Launching at MSR does \textbf{NOT} preserve option to recommit
    }
}

\marginfigrestore